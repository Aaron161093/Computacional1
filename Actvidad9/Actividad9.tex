\documentclass{article}

% set font encoding for PDFLaTeX, XeLaTeX, or LuaTeX
\usepackage{ifxetex}
\ifxetex
  \usepackage{fontspec}
\else
  \usepackage[T1]{fontenc}
  \usepackage[utf8]{inputenc}
  \usepackage{lmodern}
  \usepackage{graphicx}
  \usepackage{siunitx}
  \usepackage{amsmath}
  \graphicspath{ {images/} }
\fi

\usepackage{hyperref}
\usepackage[left=3cm,right=3cm,top=3cm,bottom=3cm]{geometry}
\title{Actividad 9}
\author{Luis Aarón Cerón Ramírez}

% Enable SageTeX to run SageMath code right inside this LaTeX file.
% http://mirrors.ctan.org/macros/latex/contrib/sagetex/sagetexpackage.pdf
% \usepackage{sagetex}

\begin{document}
\maketitle
\section{Introducción}
En esta actividad se centra en el uso de una nueva herramienta computacional, como lo es wxmaxima, el cual es un sistema de álgebra computacional Maxima es un motor de cálculo simbólico escrito en lenguaje Lisp publicado bajo licencia GNU GPL.
\newline
Cuenta con un amplio conjunto de funciones para hacer manipulación simbólica de polinomios, matrices, funciones racionales, integración, derivación, manejo de gráficos en 2D y 3D, manejo de números de coma flotante muy grandes, expansión en series de potencias y de Fourier, entre otras funcionalidades. Además tiene un depurador a nivel de fuente para el código de Maxima.
En este trabajo se hará una pequeña síntesis sobre un tutorial que nos haya parecido interesante. En este trabajo se abordara la sección de ecuaciones diferenciales.

\section{Desarrollo}
\subsection{Introducción a las ecuaciones diferenciales}
La sección describe las funciones disponibles en Maxima con las cuales se obtienen soluciones análiticas para tipos específicos de ecuaciones diferenciales de primer y segundo orden. Para solucionar sistemas de ecuaciones diferenciales se debe adicionar el paquete dynamics y para las representaciones grafícas es necesario incluir el paquete plotdf.

\subsection{Funciones y variables para ecuaciones diferenciales}
Function: bc2 (solution, xval1, yval1, xval2, yval2)
\newline
Resolver una ecuación diferencial de segundo orden con condiciones iniciales.
$solution$ is una solución general de la ecuación, encontrada por $ode2;xval1$ especifica los valores de la variable independiente en un primer punto, en la forma $x=x1$, y $yval1$ da los valores de la variable dependiente en un punto, de la forma $y= y1$. La expresión $xval2$ y $yval2$ dan los valores de estas variables en un segundo punto, usando la misma forma.
\newline
Function: desolve
\newline
\begin{verbatim}
desolve (eqn, x)
desolve ([eqn_1, ..., eqn_n], [x_1, ..., x_n])
\end{verbatim}
La función $desolve$ resuelve el sistema de ecuaciones diferenciales lineales usando transformadas de Laplace. Aqui $eqns$ son ecuaciones diferenciales en la variable dependiente $x_1, ..., x_n$.
\newline
La dependencia funcional $x_1, ..., x_n$ en la variable independiente, para el caso de x, debe ser explícitamente indicada en las variables y las derivadas. Por ejemplo la manera correcta de expresarlo es:
\begin{verbatim}
eqn_1: 'diff(f(x),x,2) = sin(x) + 'diff(g(x),x);
eqn_2: 'diff(f(x),x) + x^2 - f(x) = 2*'diff(g(x),x,2);
\end{verbatim}
y para llamar a la función $desolve$ escribimos
\begin{verbatim}
desolve([eqn_1, eqn_2], [f(x),g(x)]);
\end{verbatim}
Si conocemos las condiciones iniciales $x=0$, estas pueden ser suministrada antes de llamar $desolve$ usando $atvalue$, por ejemplo.
\begin{verbatim}
(%i1) 'diff(f(x),x)='diff(g(x),x)+sin(x);
                 d           d
(%o1)            -- (f(x)) = -- (g(x)) + sin(x)
                 dx          dx
(%i2) 'diff(g(x),x,2)='diff(f(x),x)-cos(x);
                  2
                 d            d
(%o2)            --- (g(x)) = -- (f(x)) - cos(x)
                   2          dx
                 dx
(%i3) atvalue('diff(g(x),x),x=0,a);
(%o3)                           a
(%i4) atvalue(f(x),x=0,1);
(%o4)                           1
(%i5) desolve([%o1,%o2],[f(x),g(x)]);
                  x
(%o5) [f(x) = a %e  - a + 1, g(x) =

                                                x
                                   cos(x) + a %e  - a + g(0) - 1]
(%i6) [%o1,%o2],%o5,diff;
             x       x      x                x
(%o6)   [a %e  = a %e , a %e  - cos(x) = a %e  - cos(x)]
\end{verbatim}
Si $desolve$ no puede obtener una solución , este devuelve un $false$ como respuesta.
\newline
Function: ic1 (solution, xval, yval)
\newline
Resolver una ecuación diferencial de primer orden, en esta caso $solution$ es la solución general, como es encontrada por $ode2, xval$ dando las condiciones iniciales para la variable independiente en la forma $x= x0$, y $yval$ da las condiciones iniciales para la variable dependiente en la forma $y= y0$.
\newline
Function: ic2 (solution, xval, yval, dval)
\newline
Resolver un problema con condiciones iniciales de una ecuación diferencial de segundo orden, en este caso $solution$ indica la solución general de la ecuación encontrada por $ode2$, $xval$ da el valor inicial de la variable independiente en la forma $x= x0$, $yval$ da el valor inicial de la variable dependiente en la forma $y= y0$, y $dval$ da los valores iniciales para la primer derivada de la variable dependiente con respecto a la variable independiente, en la forma $diff(y,x)= dy0$(diff hace que no tenga que ser citado)
\newline
Function: ode2 (eqn, dvar, ivar)
\newline
La función $ode2$ resuelve una ecuación diferencial ordinaria de primer y segundo orden. Esta función toma los tres argumentos: una ODE da $eqn$, la variable dependiente $dvar$, y la variable independiente $ivar$. Cuando el calculo tiene exito este da como resultado en la salida una solución explíta o implicita de la variable dependiente. \% es usado para representar la constante de integracón en el caso de una EDO de primer orden mientas que \%k1 y \%k2 se usa para denotar las constantes en el caso de segundo orden. La dependencia de la variable dependiente y la independiente no tiene que ser escrita explicítamente.
\newline
Si $ode2$ no puede obtener una solución esta devuelve un
$false$ despues de imprimir un mensaje de error. El metódo implementado para la dcuación de primer orden es probar si es: lineal, separable, exacta, homogénea, ecuación de Bernoulli y un metódo homogéneo generalizado, mientras que para las ecuaciones de segundo orden pueden ser resueltas por: coeficientes constantes, exactas, homogéneas lineales con coeficientes no constantes que pueden ser transformada a coeficientes constantes, ecuación de Euler, variación de parámetros entre otras.
\newline
Varias variables se establecen solo con fines informativos:$method$, denota el metódo de soluciones usado, $infractor$ denota cualquier factor de integración usado, $odeindex$ denota el indíce para el metódo de Bernoulli para el metódo homogéneo generalizado y $yp$ denota la solución particular cuando se utiliza variación de parámetros.
\newline
Para resolver problemas de valores iniciales (IVP) las funciones $ic1$ y $ic2$ están disponibles para ecuaciones de primer y segundo orden, y para resolver problemas de valores límite de segundo orden (BVP) se puede usar la función $bc2$.Ejemplo de la utilización de estos comandos:
\begin{verbatim}
(%i1) x^2*'diff(y,x) + 3*y*x = sin(x)/x;
                      2 dy           sin(x)
(%o1)                x  -- + 3 x y = ------
                        dx             x
(%i2) ode2(%,y,x);
                             %c - cos(x)
(%o2)                    y = -----------
                                  3
                                 x
(%i3) ic1(%o2,x=%pi,y=0);
                              cos(x) + 1
(%o3)                   y = - ----------
                                   3
                                  x
(%i4) 'diff(y,x,2) + y*'diff(y,x)^3 = 0;
                         2
                        d y      dy 3
(%o4)                   --- + y (--)  = 0
                          2      dx
                        dx
(%i5) ode2(%,y,x);
                      3
                     y  + 6 %k1 y
(%o5)                ------------ = x + %k2
                          6
(%i6) ratsimp(ic2(%o5,x=0,y=0,'diff(y,x)=2));
                             3
                          2 y  - 3 y
(%o6)                   - ---------- = x
                              6
(%i7) bc2(%o5,x=0,y=1,x=1,y=3);
                         3
                        y  - 10 y       3
(%o7)                   --------- = x - -
                            6           2
\end{verbatim}

\section{Conclusión}
En este trabajo se exploro una herramienta de mucha utilidad pues nos brinda otra opción para trabajar, en este caso se necesita aprender la sintaxís desde el principio, cosa que no da mucho problema pues se vuelve intuitiva. Wxmaxima cuanta con muchas opciones con las que se puede trabajar, en este caso escogí ecuaciones diferenciales pues es un tema que vale dominar en el lenguaje de programación pues muchos fenómenos y problemas las utilizan.
\newline
Otra ventaja de trabajar con wxmaxima es el hecho de que este es un software libre por lo que no es una opción viable para trabajar en casa, sin la necesiada de comprar la licencia como seria el caso para mapleTA ó mathematica.

\section{Apendíce}
¿Cuál fue tu primera impresión de wxmaxima?
\newline
Una buena opción cuando no se cuenta con mathematica o mapleTa
\newline
¿Crees que esta herramienta puede ser útil en otros de tus cursos?
\newline
Si, probablemente en todas ya que maxima nos brindaria una buena forma para confirmar resultados
\newline
¿Qué se te dificultó mas en esta actividad?
\newline
Traducir el texto y la sintaxís de maxima
\newline
¿Se te hizo compleja esta actividad? ¿Cómo la mejorarías? 
\newline
No me parecio compleja, tal vez mejoraria dando mas fuentes para buscar y ya que con los videos es dificil seguir la explicacion y practicar.


\end{document}
